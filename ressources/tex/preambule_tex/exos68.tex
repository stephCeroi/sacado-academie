\documentclass[10pt]{article}
%\input{preambule.tex}

\usepackage{pgfplots}
\pgfplotsset{compat=1.15}
%\usepackage{mathrsfs}
\usetikzlibrary{arrows}


\begin{document}
Chacune des paraboles $C_1$, $C_2$, $C_3$, $C_4$ construites ci-dessous est la représentation d'une fonction polynôme $f$ de degré 2 de forme canonique $f(x)=a(x-\alpha)^2+\beta$.


\definecolor{ffvvqq}{rgb}{1.,0.3333333333333333,0.}
\definecolor{qqqqff}{rgb}{0.,0.,1.}
\definecolor{ccqqqq}{rgb}{0.8,0.,0.}
\definecolor{qqwuqq}{rgb}{0.,0.39215686274509803,0.}
\begin{tikzpicture}[line cap=round,line join=round,>=triangle 45,x=1.0cm,y=1.0cm]
\begin{axis}[
x=1.0cm,y=1.0cm,
axis lines=middle,
ymajorgrids=true,
xmajorgrids=true,
xmin=-6.940000000000001,
xmax=8.180000000000003,
ymin=-3.6000000000000045,
ymax=7.299999999999999,
xtick={-6.0,-5.0,...,8.0},
ytick={-3.0,-2.0,...,7.0},]
\clip(-6.94,-3.6) rectangle (8.18,7.3);
\draw[line width=2.pt,color=qqwuqq,smooth,samples=100,domain=-3.0000000000001:4.180000000000003] plot(\x,{0-3*(\x)^(2)-18*(\x)-25});
%\draw[line width=2.pt,color=ccqqqq,smooth,samples=100,domain=-6.940000000000001:8.180000000000003] plot(\x,{0-(\x)^(2)+6*(\x)-7});
%\draw[line width=2.pt,color=qqqqff,smooth,samples=100,domain=-6.940000000000001:8.180000000000003] plot(\x,{1/3*(\x)^(2)+4/3*(\x)+18/3});
%\draw[line width=2.pt,color=ffvvqq,smooth,samples=100,domain=-6.940000000000001:8.180000000000003] plot(\x,{2*(\x)^(2)-8*(\x)+5});
\end{axis}
\end{tikzpicture}



\begin{enumerate}
\item Déterminer pour chacune des courbes les valeurs de $\alpha$, $\beta$ et $a$.
\item Associer à chaque fonction polynôme trouvé sa forme développée.
\begin{enumerate}
\item $f(x)=-3x^2-18x-25$.
\item $g(x)=-x^2+6x-7$.
\item $h(x)=\frac{1}{3}x^2+\frac{4}{3}x+\frac{18}{3}$.
\item $k(x)=2x^2-8x+5$.
\end{enumerate}
\end{enumerate}



Philippe Demaria
2 avenue de la pinède
83400 La Capte 
Hyères
France
Tel : +33 622478627

https://sacado.xyz
\end{document}
